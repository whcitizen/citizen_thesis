%This is the source code of the document "LaTeX Example".
\documentclass[english,a4paper]{jsarticle}
\usepackage{amsmath,amssymb}%used when you write numerical formula(数式)
\usepackage{bm}%used when you write bold italic character (vector)
\usepackage{float}%used when you use [H] command in figure and table
\usepackage[dvipdfmx]{graphicx}%used when you use graphics
\usepackage{tabularx}%used when you use table

\title{\LaTeX{} Example}
\author{Yuichi TSURUNO}
\date{2018/06/05}

\begin{document}
\pagestyle{empty}
\setlength{\baselineskip}{16pt}%set if you want to change linespace

\maketitle

\tableofcontents%目次

\listoffigures%図目次

\listoftables%表目次

\newpage%改ページ

\pagestyle{headings}

\section{Lists (from presentation slides)}
\subsection{Advantages}
\begin{enumerate}
	\item Various mathematical Equations are written easily.
	\item Layout is better.
	\item Table of contents(目次)and bibliography (参考文献)are produced automatically.
	\item Work speedy
\end{enumerate}

\subsection{Disadvantages}
\begin{itemize}
\item Studying commands is needed.
\item Table and coloring are bothersome.
\item Install is difficult.
\begin{itemize}
	\item By TeX Live (\LaTeX{} with many package).
\end{itemize}
\item Compile error
\end{itemize}

\section{Example of Various Function}
\subsection{Numerical Formula}
\subsubsection{Basic?}
%Verbatim is used when you display commands as they are.
\begin{verbatim}
\[\left(\int _0^\infty \frac{\sin x}{\sqrt{x}}\mathrm{dx}\right)^2
=\sum_{k=0}^\infty \frac{(2k)!}{2^{2k}(k!)^2}\frac{1}{2k+1}
=\prod_{k=1}^{\infty}\frac{4k^2}{4k^2-1}=\frac{\pi}{2}\]
\end{verbatim}
\[\left(\int _0^\infty \frac{\sin x}{\sqrt{x}}\mathrm{dx}\right)^2
=\sum_{k=0}^\infty \frac{(2k)!}{2^{2k}(k!)^2}\frac{1}{2k+1}
=\prod_{k=1}^{\infty}\frac{4k^2}{4k^2-1}=\frac{\pi}{2}\]

\[\frac{\partial u}{\partial t}
+u\frac{\partial u}{\partial x}+v\frac{\partial u}{\partial y}
+w\frac{\partial u}{\partial z}
=-\frac{1}{\rho}\frac{\partial p}{\partial x}\]

\[\frac{\partial \bm{v}}{\partial t}+\bm{v}\cdot \nabla\bm{v}
=-\frac{1}{\rho}\nabla p+\bm{g}\]

$y=ax^2+bx+c$ means parabola.

\subsubsection{Matrix}
\begin{equation}
\mathrm{det}\begin{pmatrix}
X_0'' & x' & u'' \\
Y_0'' & y' & v'' \\
Z_0'' & z' & w''
\end{pmatrix}
=
\begin{vmatrix}
X_0'' & x' & u'' \\
Y_0'' & y' & v'' \\
Z_0'' & z' & w''
\end{vmatrix}
=0
\end{equation}

\newpage

\section{Figure}
Fig. \ref{fig:logo} is as follows.
\begin{figure}[H]
	\centering
	\graphicspath{{graphics/}}
	\includegraphics[width=2cm]{logo.png}
	\caption{Logo of Survey Lab.}
	\label{fig:logo}
\end{figure}%

\section{Table}
Table \ref{table1} is as follows.
\begin{table}[H]
	\begin{center}
		\caption{An example of table (from my graduation thesis)}
		\begin{tabular}{l|rrrr}
			& OLS & cdacv5 & cdacv10 & cdacv30\\ \hline
			選択された説明変数 & 3 & 30 & 31 & 34\\
%			決定係数 & 0.173 & 0.452 & 0.458 & 0.462\\
			$R_f^2$ & 0.171 & 0.263 & 0.272 & 0.277\\
%			排除率 & 0.000 & 0.907 & 0.904 & 0.894\\
%			正則化係数 &  & 33450.683 & 29093.698 & 27132.880\\
			全説明変数 & 3 & 322 & 322 & 322\\
%			切片 & 687221.429 & 687221.429 & 687221.429 & 687221.429\\
		\end{tabular}
		\label{table1}
	\end{center}
\end{table}

\section{Sample sentences}
The goal of the International Graduate Program created by the Civil Engineering 
Department at the University of Tokyo is to prepare its graduating students 
to become future international leaders. Here, at the Civil Engineering Department, 
we are well aware of the skills, both academically and professionally, 
that are essential for success in the current global situation. Experimentation, 
exploration and involvement in key projects in various academic fields and industries 
offer students crucial experience and skills needed for developing and 
integrating competencies necessary for participation on an international level 
in our expanding global world. 
In our program, we strive to create an environment, which stimulates students 
to fully realize their individual potentials. And, the most valuable resource we offer 
is our human network, built with both current and future well recognized professionals.

\vspace{\baselineskip}%used when you use vertical space
社会基盤学は、人の生活と環境に関わる多様な専門分野を総合化し、私達の身近にあって、その暮らしを支えてきた
実践的学問体系です。基盤技術を中心に、水環境や生態系、都市問題、防災、地域や国土の計画、社会資本政策や
プロジェクトマネジメント、国際協力など、ひとつの学科にまとまるとは思えないほどのフィールドの広さを社会基盤学は
カバーしています。それらの共通点は私たちの生活基盤づくり、自然環境づくりに関わっているということに尽きると思います。
人間・自然環境の再生と創造を実現するために必要な、基盤技術・デザイン・政策決定・マネジメントなどに関する
研究・開発・実践を行うことが社会基盤学の目的です。

(Source : http://www.civil.t.u-tokyo.ac.jp/)

\end{document}